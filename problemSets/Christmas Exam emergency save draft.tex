\documentclass[12pt,letterpaper]{article}
\usepackage{graphicx,textcomp}
\usepackage{natbib}
\usepackage{setspace}
\usepackage{fullpage}
\usepackage{color}
\usepackage[reqno]{amsmath}
\usepackage{amsthm}
\usepackage{fancyvrb}
\usepackage{amssymb,enumerate}
\usepackage[all]{xy}
\usepackage{endnotes}
\usepackage{lscape}
\newtheorem{com}{Comment}
\usepackage{float}
\usepackage{hyperref}
\newtheorem{lem} {Lemma}
\newtheorem{prop}{Proposition}
\newtheorem{thm}{Theorem}
\newtheorem{defn}{Definition}
\newtheorem{cor}{Corollary}
\newtheorem{obs}{Observation}
\usepackage[compact]{titlesec}
\usepackage{dcolumn}
\usepackage{tikz}
\usetikzlibrary{arrows}
\usepackage{multirow}
\usepackage{xcolor}
\newcolumntype{.}{D{.}{.}{-1}}
\newcolumntype{d}[1]{D{.}{.}{#1}}
\definecolor{light-gray}{gray}{0.65}
\usepackage{url}
\usepackage{listings}
\usepackage{color}

\definecolor{codegreen}{rgb}{0,0.6,0}
\definecolor{codegray}{rgb}{0.5,0.5,0.5}
\definecolor{codepurple}{rgb}{0.58,0,0.82}
\definecolor{backcolour}{rgb}{0.95,0.95,0.92}

\lstdefinestyle{mystyle}{
	backgroundcolor=\color{backcolour},   
	commentstyle=\color{codegreen},
	keywordstyle=\color{magenta},
	numberstyle=\tiny\color{codegray},
	stringstyle=\color{codepurple},
	basicstyle=\footnotesize,
	breakatwhitespace=false,         
	breaklines=true,                 
	captionpos=b,                    
	keepspaces=true,                 
	numbers=left,                    
	numbersep=5pt,                  
	showspaces=false,                
	showstringspaces=false,
	showtabs=false,                  
	tabsize=2
}
\lstset{style=mystyle}
\newcommand{\Sref}[1]{Section~\ref{#1}}
\newtheorem{hyp}{Hypothesis}


\title{Exam 2 (Christmas)}
\date{Due: December 08, 2021}
\author{Applied Stats/Quant Methods 1}

\begin{document}
	\maketitle
	\section*{Instructions}
	\begin{itemize}
		\item Instructions
		Please read carefully: You have until 23:59 Wednesday December 8 to complete
		the exam. Please export your answers as a single PDF file and include all code you
		produce in a supporting R file, which you will upload to Blackboard. The exam is
		open book; you can consult any materials you like. You must not collborate with
		or seek help from other students. In case of questions or technical difficulties, you
		can contact me via email. You should write-up your answers in R and LaTeX as you
		would for a problem set. Please make sure to concisely number your answers so that
		they can be matched with the corresponding questions. 
		
		
	\end{itemize}
	\lstinputlisting[language=R, firstline=10, lastline=81]{.pdf}
	
	\newpage
	
	\vspace{.5cm}
	\section*{Question 1: Stock Market DONE}
	Suppose we were interested in studying the value of our company in the stock market. Figure
	1 plots the total value of our stock (the y-axis is in dollars).
	
	\vspace{
		\noindent 	
		
		\noindent 
		
		\begin{enumerate}
			
			\item [(a)] What concerns might we have about using the value of our company ‘as is’ in a model that
			regresses 
			‘Total Stock Value’ on ‘Months After Purchase’?
			Skewness and Kurtosis. 
			
			This appears to be an example of data that are drawn from a distribution that is right-skewed or positive skewed (in this case it appears to be the exponential distribution). 
			
			
			Another concern for this data may be excess kurtosis, either positive or negative. Underdispersed data has negative excess kurtosis and therefore a reduced number of outliers. 
			Overdispersed data has a positive excess kurtosis and an increased number of outliers.
			
			\begin{verbatim}	
			\end{verbatim}
			
			\item [(b)]  How could we address these
			concerns?
			Skew: There are apparently numerous ways to address skew in data.One way is the data can be transformed, using log and root for exponential distributions and Box Cox for skewed distributions. Outliers can also be removed manually, identifying them using z scores and interquartile range.
			
			Kurtosis: Transforming the data. The Box-Cox transformation is a useful technique for trying to normalise a data set. Another technique available is the probability plot correlation coefficient plot and the probability plot to investigate a good distributional model for the data. 
			
		\end{enumerate}
		
		\newpage
		
		\section*{Question 2: Lambs}
		This data set presents information on 33 lambs, of which 11 are ewe lambs, 11 are wether
		lambs, and 11 are ram lambs. These lambs grazed together in the same pasture and were
		treated similarly in all ways. The variables of interest for this question are stated below.
		The objective is to determine whether differences in Fatness could be attributed to Group
		while accounting for Weight. Information on the data and the model fit in R are given below:
		
		\vspace{.25cm}
		\noindent 
		
		
		\vspace{.5cm}
		
		
		\vspace{.5cm}
		\begin{enumerate}
			\item [(a)] Write out the fitted model for a ewe lamb using the estimated coefficients.
			
			
			
			\begin{verbatim}
				
			\end{verbatim}
			
			\newpage		
			\item [(b)] What is the predicted Fatness index of a wether lamb that weighs 14kg?
			
			\begin{verbatim}
				
			\end{verbatim}
			
			
			\item [(c)] Which lamb group has the highest Fatness index for every weight?
			
			
			\begin{verbatim}
				
				
			\end{verbatim}
			\vspace{.5cm}
			\section*{Question 3: Arsenic}
			Many of the wells used for drinking water in Bangladesh and other South Asian countries are
			contaminated with natural arsenic, affecting an estimated 100 million people. Arsenic is a
			cumulative poison, and exposure increases the risk of cancer and other diseases, with risks
			estimated to be proportional to exposure.
			A research team measured all the wells in one village, and labeled them with their arsenic
			level to characterise wells as “safe” (below 0.5 in units of hundreds of micrograms per liter,
			the Bangladesh standard for arsenic in drinking water) or “unsafe” (above 0.5).
			We performed a regression analysis with the data to understand the factors that predict the
			arsenic level of households’ drinking water. Your outcome variable arsenic is a continuous
			measure of household i’s arsenic level in units of hundreds of micrograms per liter.
			We estimated models with the following inputs:
			• The distance (in kilometers/100) to the closest known commercial factory
			• Depth of respondent’s well (binary variable; deep=1, not deep=0)
			\item [(a)] 
			So, we successfully estimated an additive model with well depth and distance to the
			nearest factory as the two predictors of a household’s arsenic level. The estimated
			coefficients are found in the first column of Table 1. Interpret the estimated coefficients
			for the intercept and each predictor.
			\begin{verbatim}
				
				
				
			\end{verbatim}
			\item [(b)] Does the coefficient estimate for the closest known factory vary based on whether or not
			a house has a deep well? If so, change your interpretation of the estimated coefficients
			in part (a) to conform with the interactive model in column 2 of Table 1. Provide
			the appropriate test to determine whether we should model the relationship between
			distance, well depth, and arsenic levels using an additive or interactive model.
			
			
			\begin{verbatim}
				
				
				
			\end{verbatim}
			\item [(c)] Compute the average difference in arsenic levels between two households that have a
			deep well (=1), but one is closer to a factory (dist100 = 0.4) than the other (dist100 =
			2.14).
			
			\begin{verbatim}
				
				
				
			\end{verbatim}
			\newpage
			\section*{Question 4:Multiple Choice DONE}
			\item [(a)] For explanatory variables with multi-collinearity, the corresponding estimated slopes have
			larger standard errors.
			VIFDONE .
			
			\item [(b)] The coefficients in an ordinary least squares regression model are generalized additive estimates.
			DONE 
			
			\item [(c)] We can calculate our standard errors by taking the square root of the off-diagonal elements
			in our variance-covariance matrix.
			FALSE
			diagonal elements DONE
			
			\item [(d)] Which of the following plots is used to check for normality in the assumptions of linear
			regression?
			The QQ plot of residuals.
			DONE
			
			\vspace{.5cm}
			\newpage
			\section*{Question 5: Climate Action}
			The following is a regression where the outcome measures individuals’ desire to combat
			climate change as indicated by feeling thermometer ratings (the variable ranges from 0 to 100
			where 100 indicates high levels of support for action to combat climate change). Researchers
			use three explanatory variables in their regression. First, they include a standard 7-point
			party identification measure that ranges from 1-‘Strong Republican’ to 7-‘Strong Democrat’
			(Party). Second, they include a dummy variable (0 or 1) indicating whether the respondent
			is below the age of 70, or 70 and above (Age). Last, the researchers have information on
			the number of years that respondents attended school (Education). The regression includes
			N=1194 observations.
			
			\item [(a)]  Interpret the coefficients for Age and Education
			
			
			\begin{verbatim}
				
				
				
			\end{verbatim}
			\item [(b)] ) The author claims that she ’cannot reject the null hypothesis that Age has no effect
			on support for climate action (H0 : βAge = 0) . Using the coefficient estimate and the
			standard error for Age construct a 95% confidence interval for the effect of Age on
			support for climate action. Based on the confidence interval, do you agree with the
			author? Explain your answer.
			
			\begin{verbatim}
				
				
				
			\end{verbatim}
			\item [(c)] Calculate the first difference in support for climate action between low and high values
			of Education for young respondents holding Party constant at its sample mean. Use
			3.93 as the mean of Party and use +/- one standard deviation around the mean of
			Education (from 10.99 to 12.99) for low and high values of Education respectively
			
			\begin{verbatim}
				
				
				
			\end{verbatim}
			\vspace{.5cm}
			\newpage
			\section*{Question 6: Define importance of terms DONE}
			Define and describe why the following four (4) terms are important to hypothesis testing
			and/or regression. You can earn full credit with just two or three sentences, but please be
			specific and thorough.
			\item [(a)] Partial F-test
			Used to determine whether there is a statistically significant difference between a regression model and a subset of variables/ nest version of the overall regression model. Used to test the usefulness of a group of specific predictors in the overall model i.e. improve the fit of the model.
			The null hypothesis is that all coefficients removed from the overall model are zero. The alternative is that at least one of these removed coefficients in the nest model is not zero. DONE
			
			\begin{verbatim}
			\end{verbatim}
			\item [(b)] Categorical data/dummy variables
			It does not make sense to assign values of 1,2,3, to categorical data. Decomposing categorical variables into dummy variables in this case is important, assigning values of instead 1 or 0. This allows us to use Categorical data to predict in our regression.
			Dummy variable models apparently are also important provide correct results for un/imbalanced data. DONE 
			\begin{verbatim}
			\end{verbatim}
			\item [(c)] Constituent term, alse referred to as constitutive terms, is used in relation to the interaction effect. Example of constituent or constitutive terms is the/a coefficient. The product of constitutive terms is interaction terms.
			DONE
			\begin{verbatim}
			\end{verbatim}
			\item [(d)] The Test statistic is important as it let's us know whether we can reject or fail to reject the null hypothesis test. If the standardised test statistic is more extreme than the critical value, reject. If not more extreme, fail to reject. For example, a two sample t-test  tests if the means of two populations are equal. DONE
			\begin{verbatim}
			\end{verbatim}
		\end{enumerate}  
		
		
	\end{document}
	
	
	
